\section{Risk Analysis and Security Measures}

\subsection{Assets}

\subsubsection{{\it Physical assets}}
Server - The web server is located in a virtual machine and is up to date. The administrator is responsible to install all the patches guarantying the proper functionality. 
Physical access to may allow an adversary to gain control of the system. A server can,
for example, be booted with a different operating system.

\subsubsection{{\it Logical assets}}
It includes the operating system, the website, the database, the information related to the users and customer confidence. 

Firewall - The IP table (firewall) of the server is properly configured and restricts access to the server. It keeps track of each connection passing through it and filters all the attempts to connect to the server except through allowed ports. The firewall is kept up-to-date and the administrator installs all security-relevant updates.

\subparagraph{} Website - The website provides the following functionality - uploading images, sharing them with others user, and posting comments. Only a user authorised for a picture can view, comment or read comments on that picture. No unauthorised user can prevent an image or a comment from being shown to authorised users. The website runs on the web-server which is kept up-to-date. The web developer is responsible for updating the functionality of the website. 

Database - it keeps all the information related to the users, using the website. The username, passwords, pictures are stored on the database. It runs on the server and the access to it is restricted. The administrator is responsible for its maintenance.

All the information related the users is valuable and proper measures are taken to guarantee their confidentiality. The information includes especially the username, password, pictures uploaded by users. 

Pictures - all the uploaded images are kept on the database and are visible to the owner and others allowed by the users. There is no restrictions on the size of the images.

Usernames and passwords - They identify the owner of the pictures. All password are saved, using the proper hash functions guarantying the security. 

Customer confidence - since the user can upload private pictures, which should be hidden for the world, user confidence is a important for a successful business relationship.

\subsection{Threat Sources}



\begin{itemize}
  \item \textit{Employees} : the only employee in our tiny company consist of a system administrator who has access to the server,he could possibly leak sensitive information (intentionally or unintentionally) or weaken the system security.
  \item \textit{Hackers} : since the system is connected to internet,it is exposed to various attacks,the attackers vary from highly motivated people with good skills actively trying to penetrate the system to script kidies just messing around.
  \item \textit{Malware} : as any it system, malware could possibly be a problem, it could be directed malware (unlikely) or undirected malware (more likely).
\end{itemize}



\subsection{Risks and Countermeasures}




%\subsubsection{Tools}

\begin{center}
\begin{tabular}{|l|l|}
\hline
\multicolumn{2}{|c|}{\bf Impact} \\
\hline
Impact & Description \\
\hline
\hline
High & Complete shutdown of the system, \\
     & user data is compromised, \\
     & apocalypse and a major loss in asset value,\\
     & complete loss of the customers confidence, \\
     
\hline
Medium & system slow down\\
	   & loss in asset value \\
\hline
Low   & relatively affect the credibility of the company,\\
& lower the customers confidence and a relative loss in asset value\\
\hline
\end{tabular}
%
%\vspace{5mm}
%
%\noindent \hspace*{10pt}
\begin{tabular}{|l|l|}
\hline
\multicolumn{2}{|c|}{\bf Likelihood} \\
\hline
Likelihood & Description \\
\hline
\hline
High   &  the threat source has the power to exploit vulnerabilities in the system,\\
& the countermeasures are inexistant or ineffective \\
\hline
Medium & the threat source is motivated, \\
& some countermeasures are implemented which may prevent him to do harm \\
\hline
Low   & the countermeasures are completely effective,(almost) nothing to worry about \\
\hline
\end{tabular}
\end{center}

\vspace{5mm}

\begin{center}
\begin{tabular}{|l|c|c|c|}
\hline
\multicolumn{4}{|c|}{{\bf Risk Level}} \\
\hline
{{\bf Likelihood}} & \multicolumn{3}{c|}{{\bf Impact}} \\ %\cline{2-4}
     & Low & Medium & High \\  \hline
 High & Low & Medium & High  \\
\hline
 Medium & Low & Medium & Medium \\
\hline
 Low & Low & Low & Low \\
\hline
\end{tabular}
\end{center}

\subsubsection{\it Evaluation Asset Firewall}

\begin{table}[H]
\centering
\caption{Evaluation Asset Firewall}
\label{my-label}
\begin{tabular}{|l|l|l|l|l|l|}
\hline
\multicolumn{1}{|c|}{No.} & \multicolumn{1}{c|}{Threat}                                                                                                                                    & \multicolumn{1}{c|}{\begin{tabular}[c]{@{}c@{}}Implemented/planned \\ countermeasure(s)\end{tabular}}                                 & \multicolumn{1}{c|}{L} & \multicolumn{1}{c|}{I} & \multicolumn{1}{c|}{R} \\ \hline
1                         & \begin{tabular}[c]{@{}l@{}}Malware:\\ Virus/worm spreads \\ over the Internet \\ possibly affects system \\ files and change \\ firewall settings\end{tabular} & \begin{tabular}[c]{@{}l@{}}Proper maintenance of the \\ server, security patches \\ installed, restricted \\ user rights\end{tabular} & Low                    & Medium                 & Low                    \\ \hline
\end{tabular}
\end{table}


\subsubsection{\it Evaluation Asset Website}

\begin{table}[H]
\centering
\caption{Evaluation Asset Website}
\label{my-label}
\begin{tabular}{|l|l|l|l|l|l|}
\hline
\multicolumn{1}{|c|}{No.} & \multicolumn{1}{c|}{Threat}                                                                                                                                                                                                        & \multicolumn{1}{c|}{\begin{tabular}[c]{@{}c@{}}Implemented/planned \\ countermeasure(s)\end{tabular}}                                                                                 & \multicolumn{1}{c|}{L} & \multicolumn{1}{c|}{I} & \multicolumn{1}{c|}{R} \\ \hline
1                         & \begin{tabular}[c]{@{}l@{}}Skilled hacker gains \\ control\\ over the \\ website, steals confidential \\ data, modifies website \\ settings because of \\ vularability either in \\ the server\\ or on \\ the website\end{tabular} & \begin{tabular}[c]{@{}l@{}}The\\ server is \\ hardened and regularly \\ updated. System \\ administrators are \\ trained to notice \\ irregularities on\\ the \\ server.\end{tabular} & Low                    & Medium                 & Low                    \\ \hline
2                         & \begin{tabular}[c]{@{}l@{}}Script\\ kiddies makes \\ modifications on the\\  website as a result \\ of an attack\end{tabular}                                                                                                      & \begin{tabular}[c]{@{}l@{}}The\\ server is \\ properly maintained, \\ the website is \\ hardened, all input \\ is sanitized\end{tabular}                                              & Low                    & Medium                 & Low                    \\ \hline
\end{tabular}
\end{table}


\subsubsection{\it Evaluation Asset Database}
\begin{table}[H]
\centering
\caption{Evaluation Asset Database}
\label{my-label}
\begin{tabular}{|l|l|l|l|l|l|}
\hline
\multicolumn{1}{|c|}{No.} & \multicolumn{1}{c|}{Threat}                                                                                                                                                                               & \multicolumn{1}{c|}{\begin{tabular}[c]{@{}c@{}}Implemented/planned \\ countermeasure(s)\end{tabular}}                      & \multicolumn{1}{c|}{L} & \multicolumn{1}{c|}{I} & \multicolumn{1}{c|}{R} \\ \hline
1                         & \begin{tabular}[c]{@{}l@{}}Skilled hacker gains \\ control,over the \\ database, steals \\ confidential data, \\ make,changes on \\ the database like \\ deleting tables, \\ editing records\end{tabular} & \begin{tabular}[c]{@{}l@{}}Hardened\\ server and \\ kept up-to-date, \\ use of hashed \\ and salted passwords\end{tabular} & Low                    & High                   & Low                    \\ \hline
\end{tabular}
\end{table}


\subsubsection{\it Evaluation Asset Images}
\begin{table}[H]
\centering
\caption{Evaluation Asset Images}
\label{my-label}
\begin{tabular}{|l|l|l|l|l|l|}
\hline
\multicolumn{1}{|c|}{No.} & \multicolumn{1}{c|}{Threat}                                                                                                            & \multicolumn{1}{c|}{\begin{tabular}[c]{@{}c@{}}Implemented/planned \\ countermeasure(s)\end{tabular}} & \multicolumn{1}{c|}{L} & \multicolumn{1}{c|}{I} & \multicolumn{1}{c|}{R} \\ \hline
1                         & \begin{tabular}[c]{@{}l@{}}Web developer \\ unintentionally breaks \\ confidentiality during \\ the update of the website\end{tabular} & Well trained web developer                                                                            & Low                    & High                   & Low                    \\ \hline
\end{tabular}
\end{table}


\subsubsection{\it Evaluation Asset Username and Password}
\begin{table}[H]
\centering
\caption{Evaluation Asset Username and Password}
\label{my-label}
\begin{tabular}{|l|l|l|l|l|l|}
\hline
\multicolumn{1}{|c|}{No.} & \multicolumn{1}{c|}{Threat}                                                                                   & \multicolumn{1}{c|}{\begin{tabular}[c]{@{}c@{}}Implemented/planned \\ countermeasure(s)\end{tabular}}                       & \multicolumn{1}{c|}{L} & \multicolumn{1}{c|}{I} & \multicolumn{1}{c|}{R} \\ \hline
1                         & \begin{tabular}[c]{@{}l@{}}Script kiddies try to \\ guess the username \\ and the password\end{tabular}       & \begin{tabular}[c]{@{}l@{}}Advising users not \\ to choose simple \\ usernames\\ and passwords\end{tabular}                 & Medium                 & Medium                 & Medium                 \\ \hline
2                         & \begin{tabular}[c]{@{}l@{}}Skilled hacker attacks \\ with special software \\ to break passwords\end{tabular} & \begin{tabular}[c]{@{}l@{}}Encryption\\ passwords \\ with strong hash \\ functions choosing \\ arbitrary salts\end{tabular} & Medium                 & High                   & Medium                 \\ \hline
\end{tabular}
\end{table}

\subsubsection{\it Evaluation Asset Customer confidence}
\begin{table}[H]
\centering
\caption{Evaluation Asset Customer confidence}
\label{my-label}
\begin{tabular}{|l|l|l|l|l|l|}
\hline
\multicolumn{1}{|c|}{No.} & \multicolumn{1}{c|}{Threat}                                                         & \multicolumn{1}{c|}{\begin{tabular}[c]{@{}c@{}}Implemented/planned \\ countermeasure(s)\end{tabular}} & \multicolumn{1}{c|}{L} & \multicolumn{1}{c|}{I} & \multicolumn{1}{c|}{R} \\ \hline
1                         & \begin{tabular}[c]{@{}l@{}}Theft of confidential \\ data\end{tabular}               & \begin{tabular}[c]{@{}l@{}}State-of-the-art \\ security measures\end{tabular}                         & Low                    & High                   & Low                    \\ \hline
2                         & \begin{tabular}[c]{@{}l@{}}Customer unsatisfied \\ of the web services\end{tabular} & Attractive design                                                                                     & Medium                 & High                   & Medium                 \\ \hline
\end{tabular}
\end{table}

\subsubsection{Detailed Description of Selected Countermeasures}

Optionally explain the details of the countermeasures mentioned above.



\subsubsection{Risk Acceptance}

List all medium and high risks, according to the evaluation above. For each risk, propose additional countermeasures that could be implemented to further reduce the risks.

\begin{footnotesize}
\begin{prettytablex}{p{2cm}X}
No. of threat & Proposed countermeasure including expected impact  \\
\hline
x & Allowing users only 3 login attempts \\
\hline
y & using one time password sent by sms \\
\hline
\end{prettytablex}
\end{footnotesize}

\end{document}

%%% Local Variables: 
%%% mode: latex
%%% TeX-master: "../../book"
%%% End: 
