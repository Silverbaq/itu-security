\section{Risk Analysis and Security Measures}

\subsection{Assets}

Describe the relevant assets and their required security
  properties. For example, data objects, access restrictions,
  configurations, etc.

\subsection{Threat Sources}
\begin{enumerate}

\item \textit{Employees} : the only employee in our tiny company consist of a system administrator (Steffen) who has access to the server,he could possibly leak sensitive information intentionally or unintentionally.
\item \textit{Hackers} : since the system is connected to internet,it is exposed to various attacks,the attackers vary from highly motivated people with good skills actively trying to penetrate the system to script kidies just messing around.
\item \textit{Malware} : as any it system, malware could possibly be a problem, it could be directed malware (unlikely) or undirected malware (more likely).
\end{enumerate}
 \subsection{Risks and Countermeasures}
\subsection{Risks and Countermeasures}

List all potential threats and the
  corresponding countermeasures. Estimate the risk based on 
  the information about the threat, the threat sources and the 
  corresponding countermeasure. For this purpose, use the following three
  tables.

%\subsubsection{Tools}

\begin{center}
\begin{tabular}{|l|l|}
\hline
\multicolumn{2}{|c|}{\bf Impact} \\
\hline
Impact & Description \\
\hline
\hline
High & Complete shutdown of the system, \\
     & user data is compromised, \\
     & apocalypse and a major loss in asset value,\\
     & complete loss of the customers confidence, \\
     
\hline
Medium & High response time \\
\hline
Low   & the system is temporary out of service\\
\hline
\end{tabular}
%
%\vspace{5mm}
%
%\noindent \hspace*{10pt}
\begin{tabular}{|l|l|}
\hline
\multicolumn{2}{|c|}{\bf Likelihood} \\
\hline
Likelihood & Description \\
\hline
\hline
High   & \hspace*{20pt}\ldots \\
\hline
Medium & \hspace*{20pt}\ldots \\
\hline
Low   & \hspace*{20pt}\ldots \\
\hline
\end{tabular}
\end{center}

\vspace{5mm}

\begin{center}
\begin{tabular}{|l|c|c|c|}
\hline
\multicolumn{4}{|c|}{{\bf Risk Level}} \\
\hline
{{\bf Likelihood}} & \multicolumn{3}{c|}{{\bf Impact}} \\ %\cline{2-4}
     & Low & Medium & High \\  \hline
 High & Low & Medium & High  \\
\hline
 Medium & Low & Medium & Medium \\
\hline
 Low & Low & Low & Low \\
\hline
\end{tabular}
\end{center}

\subsubsection{{\it Evaluation Asset X}}

Evaluate the likelihood, impact and the resulting risk,  after implementation of the corresponding countermeasures. For each threat, clearly name the threat source and the the threat action.

\begin{footnotesize}
\begin{prettytablex}{lXp{3.5cm}lll}
No. & Threat & Implemented /planned countermeasure(s) & L & I & Risk \\
\hline
1 & release of personal information  &  using hash and salted password & {\it Low} & {\it Low} & {\it Low} \\
\hline
2 & ... & ...& {\it Medium} & {\it High} & {\it Medium} \\
\hline
\end{prettytablex}
\end{footnotesize}



\subsubsection{{\it Evaluation Asset y}}

\begin{footnotesize}
\begin{prettytablex}{lXp{6.5cm}lll}
No. & Threat & Implemented/planned countermeasure(s) & L & I & Risk \\
\hline
1 & ... & ... & {\it Low} & {\it Low} & {\it Low} \\
\hline
2 & ... & ...& {\it Medium} & {\it High} & {\it Medium} \\
\hline
\end{prettytablex}
\end{footnotesize}

\subsubsection{Detailed Description of Selected Countermeasures}

Optionally explain the details of the countermeasures mentioned above.



\subsubsection{Risk Acceptance}

List all medium and high risks, according to the evaluation above. For each risk, propose additional countermeasures that could be implemented to further reduce the risks.

\begin{footnotesize}
\begin{prettytablex}{p{2cm}X}
No. of threat & Proposed countermeasure including expected impact  \\
\hline
... & ... \\
\hline
... & ... \\
\hline
\end{prettytablex}
\end{footnotesize}


