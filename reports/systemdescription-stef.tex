\documentclass{article}

\usepackage{graphicx}
\usepackage{alltt}
\usepackage{url}
\usepackage{tabularx}
%\usepackage{ngerman}
\usepackage{longtable}
\usepackage{color}

\newenvironment{prettytablex}[1]{\vspace{0.3cm}\noindent\tabularx{\linewidth}{@{\hspace{\parindent}}#1@{}}}{\endtabularx\vspace{0.3cm}}
\newenvironment{prettytable}{\prettytablex{l X}}{\endprettytablex}



\title{\huge\sffamily\bfseries System Description and Risk Analysis}
\author{w \and x \and y \and z}
\date{\dots}


\begin{document}
\maketitle

%% please observe the page limit; comment or remove lines below before hand-in
%\begin{center}
%{\large\textcolor{red}{Page limit: 30 pages.}}
%\end{center}
%%%%%%%%%%%%%%%%%%%%%%%%%%%%%%%%%%%%%%%%%%%%%%

\tableofcontents
\pagebreak


\section{System Characterization}

\subsection{System Overview}

The mission for the server is to host a web application, where users can upload there images, and share them with each other. The users has the power to choose who they wish to share there images with, and also if they would like to unshare an image with another user. When an user uploads an image, we own that image, therefor the user cannot delete or remove the image from the web application. \\

The server is a Unix-based system, build on Ubuntu 14.04. It has a bare minimum of users on the server, since there should only be one for maintaining the services running on the system.\\


This web application, is set to start-up as soon as the server is booted. This is expected to be the best case, since the main purpose of the service, is to host this application. So in case of failure where the system reboots (E.g. power out, or a like), it will be available again as soon as possible, without having personal to start up the application manually.\\

The server is also setup so that it can be accessed remotely. This is for the reason of maintenance, and will require the proper use of login and password for gaining access.
For sending files to and from the server, this should happen within the encrypted connection services running on the server, and not by additional software.


\subsection{System Functionality}

As described in the project description, the system implements the main requirements, namely:

\begin{enumerate}
\item upload pictures 
\item the user can share his own pictures with other named users on a picture-by-picture basis
\item the user can view his own pictures and pictures other has shared with him
\item the user can comment on any picture he can view
\item and the user can view comments on any picture he can view  
\end{enumerate}

    

\subsection{Components and Subsystems}

\begin{itemize}
\item Web Application: The web application is build in Python 2.7, together with the micro-framework Flask. Hence, the web server is SimpleHTTPServer, that is a part of the python standard library.

\item Database: SQLite3 is used as storage of users for the web application, and all the content that is in addition with it. This means that there is no daemon running, since everything is stored in the same file. To access and manipulation of the database, it is needed to have access to this file. This means that for altering, 

\item SSH: Open-SSH is installed on the system, this is so it is possible for a system administrator to connect to the system and do any configuration that is necessary, without having to be in front of the machine. This also means that scp is available for uploading and downloading files to the server. 
\end{itemize}

\subsection{Interfaces}

Specify  all interfaces and  information flows, from the technical as well as from the
  organizational point of view.

\subsection{Backdoors}

Easy to find: 
Netcat is running on port 60606. This makes it possible to make remote access to the service through this port.\\

The reason for putting the port number in the higher numbers, is first of all that the first 1024 ports are “well-known” ports.  Meaning that they have been defined to serve a purpose. E.g. 80 is HTTP, 443 is HTTPS, 666 is Doom. If the user makes a nmap scan on the system, these are the port the scanner will look for by default. This meaning that if someone wish to find this backdoor, they will need to specify the search a bit.\\

To find that this is the case, you can do several things.
\begin{itemize}
\item Nmap: To find this with nmap 
Command: nmap -p “*”
\item	Netstat: With netstat, it is possiable to see all the activity 
Command:
\end{itemize}

Hard to find: 
\begin{itemize}
\item Ubuntu 14.04 has an overlayfs vulnerability (CVE-2015-8660). The code to use for this exploit can be found at: https://www.exploit-db.com/exploits/39166/ - This will give local root access.
\end{itemize}

To perform the exploit, do the following:
We have compiled the exploit, and placed it under the /bin/pwn. This means that if someone on the system calls the command “pwn” from anywhere on the system, the user will gain root access. \\

For the fun of it, we have made a user with a low amount of privileges on the system. The username is admin, which have been given a password from the top 500 most used passwords. It should therefore be fairly easy to bruteforce the password. \\

So if/when this user has been cracked, someone can login to the server with SSH, call the command pwn – And bingo! Root access…\\*

Something to note:
With the web application, it is possible to upload files, as long as the file extension is of an image sort. This means that it is possible to upload any kind of data to the server since it’s only the naming of the file that makes a difference, and not the content of the file. 


\subsection{Additional Material}

You may have additional sections according to your needs.


\section{Risk Analysis and Security Measures}

\subsection{Assets}

Describe the relevant assets and their required security
  properties. For example, data objects, access restrictions,
  configurations, etc.

\subsection{Threat Sources}

Name and describe potential threat sources including their motivation.

\subsection{Risks and Countermeasures}

List all potential threats and the
  corresponding countermeasures. Estimate the risk based on 
  the information about the threat, the threat sources and the 
  corresponding countermeasure. For this purpose, use the following three
  tables.

%\subsubsection{Tools}

\begin{center}
\begin{tabular}{|l|l|}
\hline
\multicolumn{2}{|c|}{\bf Impact} \\
\hline
Impact & Description \\
\hline
\hline
High   & \hspace*{20pt}\ldots \\
\hline
Medium & \hspace*{20pt}\ldots \\
\hline
Low   & \hspace*{20pt}\ldots \\
\hline
\end{tabular}
%
%\vspace{5mm}
%
%\noindent \hspace*{10pt}
\begin{tabular}{|l|l|}
\hline
\multicolumn{2}{|c|}{\bf Likelihood} \\
\hline
Likelihood & Description \\
\hline
\hline
High   & \hspace*{20pt}\ldots \\
\hline
Medium & \hspace*{20pt}\ldots \\
\hline
Low   & \hspace*{20pt}\ldots \\
\hline
\end{tabular}
\end{center}

\vspace{5mm}

\begin{center}
\begin{tabular}{|l|c|c|c|}
\hline
\multicolumn{4}{|c|}{{\bf Risk Level}} \\
\hline
{{\bf Likelihood}} & \multicolumn{3}{c|}{{\bf Impact}} \\ %\cline{2-4}
     & Low & Medium & High \\  \hline
 High & Low & Medium & High  \\
\hline
 Medium & Low & Medium & Medium \\
\hline
 Low & Low & Low & Low \\
\hline
\end{tabular}
\end{center}

\subsubsection{{\it Evaluation Asset X}}

Evaluate the likelihood, impact and the resulting risk,  after implementation of the corresponding countermeasures. For each threat, clearly name the threat source and the the threat action.

\begin{footnotesize}
\begin{prettytablex}{lXp{6.5cm}lll}
No. & Threat & Implemented/planned countermeasure(s) & L & I & Risk \\
\hline
1 & ... & ... & {\it Low} & {\it Low} & {\it Low} \\
\hline
2 & ... & ...& {\it Medium} & {\it High} & {\it Medium} \\
\hline
\end{prettytablex}
\end{footnotesize}



\subsubsection{{\it Evaluation Asset y}}

\begin{footnotesize}
\begin{prettytablex}{lXp{6.5cm}lll}
No. & Threat & Implemented/planned countermeasure(s) & L & I & Risk \\
\hline
1 & ... & ... & {\it Low} & {\it Low} & {\it Low} \\
\hline
2 & ... & ...& {\it Medium} & {\it High} & {\it Medium} \\
\hline
\end{prettytablex}
\end{footnotesize}

\subsubsection{Detailed Description of Selected Countermeasures}

Optionally explain the details of the countermeasures mentioned above.



\subsubsection{Risk Acceptance}

List all medium and high risks, according to the evaluation above. For each risk, propose additional countermeasures that could be implemented to further reduce the risks.

\begin{footnotesize}
\begin{prettytablex}{p{2cm}X}
No. of threat & Proposed countermeasure including expected impact  \\
\hline
... & ... \\
\hline
... & ... \\
\hline
\end{prettytablex}
\end{footnotesize}

\end{document}

%%% Local Variables: 
%%% mode: latex
%%% TeX-master: "../../book"
%%% End: 
