\documentclass{article}

\usepackage{graphicx}
\usepackage{alltt}
\usepackage{url}
\usepackage{tabularx}
%\usepackage{ngerman}
\usepackage{longtable}
\usepackage{color}

\newenvironment{prettytablex}[1]{\vspace{0.3cm}\noindent\tabularx{\linewidth}{@{\hspace{\parindent}}#1@{}}}{\endtabularx\vspace{0.3cm}}
\newenvironment{prettytable}{\prettytablex{l X}}{\endprettytablex}



\title{\huge\sffamily\bfseries System Description and Risk Analysis}
\author{w \and x \and y \and z}
\date{\dots}


\begin{document}
\maketitle

%% please observe the page limit; comment or remove lines below before hand-in
%\begin{center}
%{\large\textcolor{red}{Page limit: 30 pages.}}
%\end{center}
%%%%%%%%%%%%%%%%%%%%%%%%%%%%%%%%%%%%%%%%%%%%%%

\tableofcontents
\pagebreak


\section{System Characterization}

\subsection{System Overview}

Describe the system's mission,  the system boundaries,
and the overall system architecture, including the main subsystems and
their relationships.   This description should provide a high-level
overview of the system, e.g., suitable for managers, that complements
the more technical description that follows.Hi this is a report



\subsection{System Functionality}

As described in the project description, the system implements the main requirements, namely:

\begin{enumerate}
\item the system should allow the user to upload pictures 
\item the user can share his own pictures with other named users on a picture-by-picture basis
\item the user can view his own pictures and pictures other has shared with him
\item the user can comment on any picture he can view
\item and the user can view comments on any picture he can view  
\end{enumerate}


    

\subsection{Components and Subsystems}
The system is based on the Flask micro framework which is written in Python. Flask is using Jinja to generate html which is sanitizing the fields that comes from the outside world. 
(List all system components, subdivided, for example, into
  categories such as platforms, applications, data records, etc. For
  each component, state its relevant properties.)


\subsection{Interfaces}

Specify  all interfaces and  information flows, from the technical as well as from the
  organizational point of view.

\subsection{Backdoors}

Describe the implemented backdoors. {\bfseries Do not add
    this section to the version of your report that is handed over to
    the team that reviews your system!}

\subsection{Additional Material}

You may have additional sections according to your needs.


\section{Risk Analysis and Security Measures}

\subsection{Assets}

Describe the relevant assets and their required security
  properties. For example, data objects, access restrictions,
  configurations, etc.

\subsection{Threat Sources}

Name and describe potential threat sources including their motivation.

\subsection{Risks and Countermeasures}

List all potential threats and the
  corresponding countermeasures. Estimate the risk based on 
  the information about the threat, the threat sources and the 
  corresponding countermeasure. For this purpose, use the following three
  tables.

%\subsubsection{Tools}

\begin{center}
\begin{tabular}{|l|l|}
\hline
\multicolumn{2}{|c|}{\bf Impact} \\
\hline
Impact & Description \\
\hline
\hline
High   & The system does not respond and or user data is compromised \\
\hline
Medium & High response time \\
\hline
Low   & the system is temporary out of service\\
\hline
\end{tabular}
%
%\vspace{5mm}
%
%\noindent \hspace*{10pt}
\begin{tabular}{|l|l|}
\hline
\multicolumn{2}{|c|}{\bf Likelihood} \\
\hline
Likelihood & Description \\
\hline
\hline
High   & \hspace*{20pt}\ldots \\
\hline
Medium & \hspace*{20pt}\ldots \\
\hline
Low   & \hspace*{20pt}\ldots \\
\hline
\end{tabular}
\end{center}

\vspace{5mm}

\begin{center}
\begin{tabular}{|l|c|c|c|}
\hline
\multicolumn{4}{|c|}{{\bf Risk Level}} \\
\hline
{{\bf Likelihood}} & \multicolumn{3}{c|}{{\bf Impact}} \\ %\cline{2-4}
     & Low & Medium & High \\  \hline
 High & Low & Medium & High  \\
\hline
 Medium & Low & Medium & Medium \\
\hline
 Low & Low & Low & Low \\
\hline
\end{tabular}
\end{center}

\subsubsection{{\it Evaluation Asset X}}

Evaluate the likelihood, impact and the resulting risk,  after implementation of the corresponding countermeasures. For each threat, clearly name the threat source and the the threat action.

\begin{footnotesize}
\begin{prettytablex}{lXp{3.5cm}lll}
No. & Threat & Implemented /planned countermeasure(s) & L & I & Risk \\
\hline
1 & release of personal information  &  using hash and salted password & {\it Low} & {\it Low} & {\it Low} \\
\hline
2 & ... & ...& {\it Medium} & {\it High} & {\it Medium} \\
\hline
\end{prettytablex}
\end{footnotesize}



\subsubsection{{\it Evaluation Asset y}}

\begin{footnotesize}
\begin{prettytablex}{lXp{6.5cm}lll}
No. & Threat & Implemented/planned countermeasure(s) & L & I & Risk \\
\hline
1 & ... & ... & {\it Low} & {\it Low} & {\it Low} \\
\hline
2 & ... & ...& {\it Medium} & {\it High} & {\it Medium} \\
\hline
\end{prettytablex}
\end{footnotesize}

\subsubsection{Detailed Description of Selected Countermeasures}

Optionally explain the details of the countermeasures mentioned above.



\subsubsection{Risk Acceptance}

List all medium and high risks, according to the evaluation above. For each risk, propose additional countermeasures that could be implemented to further reduce the risks.

\begin{footnotesize}
\begin{prettytablex}{p{2cm}X}
No. of threat & Proposed countermeasure including expected impact  \\
\hline
... & ... \\
\hline
... & ... \\
\hline
\end{prettytablex}
\end{footnotesize}

\end{document}

%%% Local Variables: 
%%% mode: latex
%%% TeX-master: "../../book"
%%% End: 
