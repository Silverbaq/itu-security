\documentclass{article}

\usepackage{graphicx}
\usepackage{alltt}
\usepackage{url}
\usepackage{tabularx}
\usepackage[utf8]{inputenc}
\usepackage{longtable}
\usepackage{color}
\usepackage{float}

\newenvironment{prettytablex}[1]{\vspace{0.3cm}\noindent\tabularx{\linewidth}{@{\hspace{\parindent}}#1@{}}}{\endtabularx\vspace{0.3cm}}
\newenvironment{prettytable}{\prettytablex{l X}}{\endprettytablex}



\title{\huge\sffamily\bfseries System Description and Risk Analysis}
 
\author{Malik Bekkouche \and Oscar Felipe Toro \and Steffen Mogensen \and  Yumer Adem Yumer}
\date{\today}


\begin{document}
\maketitle

%% please observe the page limit; comment or remove lines below before hand-in
%\begin{center}
%{\large\textcolor{red}{Page limit: 30 pages.}}
%\end{center}
%%%%%%%%%%%%%%%%%%%%%%%%%%%%%%%%%%%%%%%%%%%%%%

\tableofcontents
\pagebreak


\section{System Characterization}

\subsection{System Overview}
The mission for the server is to host a web application, where users can upload there images, and share them with each other. The users has the power to choose who they wish to share there images with, and also if they would like to unshare an image with another user. When an user uploads an image, we own that image, therefore the user cannot delete or remove the image from the web application. \\

The server is a Unix-based system, build on Ubuntu 14.04. It has a bare minimum of users on the server, since there should only be one for maintaining the services running on the system.\\


This web application, is set to start-up as soon as the server is booted. This is expected to be the best case, since the main purpose of the service, is to host this application. So in case of failure where the system reboots (E.g. power out, or a like), it will be available again as soon as possible, without having personal to start up the application manually.\\

The server is also set-up so that it can be accessed remotely. This is for maintenance reasons, and will require the proper use of login and password for gaining access.
For sending files to and from the server, this should happen within the encrypted connection services running on the server, and not by additional software.


\subsection{System Functionality}

As originally stated in the project description, the system implements the main requirements, namely:

\begin{enumerate}
\item the system should allow the user to upload pictures 
\item the user can share his own pictures with other named users on a picture-by-picture basis
\item the user can view his own pictures and pictures other has shared with him
\item the user can comment on any picture he can view
\item and the user can view comments on any picture he can view  
\end{enumerate}


\subsection{Components and Subsystems}

We recognize mainly two elements of the system, the web application  and the database. Both sitting on top of the Ubuntu Server. In order to connect to the server we are using the OpenSHH suite.
  
\begin{itemize}
\item Platform: the virtual machine is running Ubuntu Server 14.04.01 which is the Long Term Support version of the server.


\item Web Application: The web application is build in Python 2.7, together with the micro-framework Flask. Hence, the web server is \\SimpleHTTPServer, that is a part of the python standard library. The framework offer inbuilt security features, such as sanitisation of fields and it only offers the minimum building blocks which http server, micro framework, only the minimum 

\item Database: SQLite3 is used as storage of users for the web application, and all the content that is in addition with it. This means that there is no daemon running, since everything is stored in the same file. To access and manipulation of the database, it is needed to have access to this file. This means that for altering, 
\end{itemize}
\begin{itemize}



\item SSH: Open-SSH is installed on the system, this is so it is possible for a system administrator to connect to the system and do any configuration that is necessary, without having to be in front of the machine. This also means that scp is available for uploading and downloading files to the server. 
\end{itemize}

\subsection{Interfaces}

Specify  all interfaces and  information flows, from the technical as well as from the
  organizational point of view.

\subsection{Backdoors}
Easy to find: 
Netcat is running on port 60606. This makes it possible to make remote access to the service through this port.\\

The reason for putting the port number in the higher numbers, is first of all that the first 1024 ports are “well-known” ports.  Meaning that they have been defined to serve a purpose. E.g. 80 is HTTP, 443 is HTTPS, 666 is Doom. If the user makes a nmap scan on the system, these are the port the scanner will look for by default. This meaning that if someone wish to find this backdoor, they will need to specify the search a bit.\\

To find that this is the case, you can do several things.
\begin{itemize}
\item Nmap: To find this with nmap 
Command: nmap -p “*”
\item	Netstat: With netstat, it is possiable to see all the activity 
Command:
\end{itemize}

Hard to find: 
\begin{itemize}
\item Ubuntu 14.04 has an overlayfs vulnerability (CVE-2015-8660). The code to use for this exploit can be found at: https://www.exploit-db.com/exploits/39166/ - This will give local root access.
\end{itemize}

To perform the exploit, do the following:
We have compiled the exploit, and placed it under the /bin/pwn. This means that if someone on the system calls the command “pwn” from anywhere on the system, the user will gain root access. \\

For the fun of it, we have made a user with a low amount of privileges on the system. The username is admin, which have been given a password from the top 500 most used passwords. It should therefore be fairly easy to bruteforce the password. \\

So if/when this user has been cracked, someone can login to the server with SSH, call the command pwn – And bingo! Root access…\\*

Something to note:
With the web application, it is possible to upload files, as long as the file extension is of an image sort. This means that it is possible to upload any kind of data to the server since it’s only the naming of the file that makes a difference, and not the content of the file. 



\subsection{Additional Material}

You may have additional sections according to your needs.


\section{Risk Analysis and Security Measures}

\subsection{Assets}

\subsubsection{{\it Physical assets}}
Server - The web server is located on a virtual machine and is up to date. The administrator is responsible of installing all the patches guarantying the proper function of the server. 
Physical access to it may allow an adversary to gain control of the system. A server can,
for example, be booted with a different operating system.

\subsubsection{{\it Logical assets}}
It includes the operating system, the website, the database, the information related to the users and customer confidence. 

\textit{Firewall} - The IP table (firewall) of the server is properly configured and restricts access to the server. It keeps track of each connection passing through it and filters all the attempts to connect to the server except through allowed ports. The firewall is kept up-to-date and the administrator installs all security-relevant updates.


\subparagraph{} \textit{Website} - The website provides the following functionality - uploading images, sharing them with others user, and posting comments. Only a user authorised for a picture can view, comment or read comments on that picture. No unauthorised user can prevent an image or a comment from being shown to authorised users. The website runs on the web-server which is kept up-to-date. The web developer is responsible for updating the functionality of the website. 

\textit{Database} - it keeps all the information related to the users, using the website. The username, passwords, pictures are stored in the database. It runs on the server and the access to it is restricted. The administrator is responsible for its maintenance.

\textit{Information} - all the informations related the users are valuable and proper measures are taken to guarantee their confidentiality. The informations include especially the username, password, pictures uploaded by users. 


\textit{Pictures} - all the uploaded images are kept on the server and are visible to the owner and other users allowed by the owner. There is no restrictions on the size of the images.


\textit{Username and passwords} - They identify the owner of the pictures. All password are saved, using the proper hash functions guarantying the security. 


\textit{Customer confidence} - since the user can upload private pictures, which should be hidden for the world, user confidence is important for a successful business relationship.

\subsection{Threat Sources}



\begin{itemize}
  \item \textit{Employees} : the employees in our tiny company consist of a system administrator who has access to the server and a web developer, they could possibly leak sensitive information (intentionally or unintentionally) or weaken the system security, 
  \item \textit{Hackers} : since the system is connected to internet,it is exposed to various attacks,the attackers vary from highly motivated people with good skills actively trying to penetrate the system to script kidies just messing around.
  \item \textit{Malware} : as any it system, malware could possibly be a problem, it could be directed malware (unlikely) or undirected malware (more likely).
\end{itemize}



\subsection{Risks and Countermeasures}

%\subsubsection{Tools}

\begin{table}[H]
\centering
\label{my-label}
\begin{tabular}{|l|l|}
\hline
\multicolumn{2}{|c|}{\textbf{Impact}}                                                                                                                                                                                        \\ \hline
\multicolumn{1}{|c|}{Impact} & Description                                                                                                                                                                                   \\ \hline
High                         & \begin{tabular}[c]{@{}l@{}}Complete shutdown of the system, \\ user data is compromised,apocalypse and a major loss \\ in asset value, complete loss of the customers confidence\end{tabular} \\ \hline
Medium                       & \begin{tabular}[c]{@{}l@{}}System slow down, \\ loss in asset value\end{tabular}                                                                                                              \\ \hline
Low                          & \begin{tabular}[c]{@{}l@{}}Relatively affect the credibility \\ of the company,lower the customers \\ confidence and a relative loss in asset value\end{tabular}                              \\ \hline
\end{tabular}
\end{table}
%
%\vspace{5mm}
%
\begin{table}[H]
\centering
\label{my-label}
\begin{tabular}{|l|l|}
\hline
\multicolumn{2}{|c|}{\textbf{Likelihood}}                                                                                                                                                                     \\ \hline
\multicolumn{1}{|c|}{Likelihood} & Description                                                                                                                                                                \\ \hline
High                             & \begin{tabular}[c]{@{}l@{}}The threat source has the power to exploit \\ vulnerabilities in the system, the countermeasures \\ are inexistant or ineffective.\end{tabular} \\ \hline
Medium                           & \begin{tabular}[c]{@{}l@{}}The threat source is motivated,some countermeasures \\ are implemented which may prevent him to do harm\end{tabular}                            \\ \hline
Low                              & \begin{tabular}[c]{@{}l@{}}The countermeasures are completely effective, \\ (almost) nothing to worry about\end{tabular}                                                   \\ \hline
\end{tabular}
\end{table}

\vspace{5mm}

\begin{center}
\begin{tabular}{|l|c|c|c|}
\hline
\multicolumn{4}{|c|}{{\bf Risk Level}} \\
\hline
{{\bf Likelihood}} & \multicolumn{3}{c|}{{\bf Impact}} \\ %\cline{2-4}
     & Low & Medium & High \\  \hline
 High & Low & Medium & High  \\
\hline
 Medium & Low & Medium & Medium \\
\hline
 Low & Low & Low & Low \\
\hline
\end{tabular}
\end{center}

\subsubsection{\it Evaluation Asset Firewall}

\begin{table}[H]
\centering
\caption{Evaluation Asset Firewall}
\label{my-label}
\begin{tabular}{|l|l|l|l|l|l|}
\hline
\multicolumn{1}{|c|}{No.} & \multicolumn{1}{c|}{Threat}                                                                                                                                    & \multicolumn{1}{c|}{\begin{tabular}[c]{@{}c@{}}Implemented/planned \\ countermeasure(s)\end{tabular}}                                 & \multicolumn{1}{c|}{L} & \multicolumn{1}{c|}{I} & \multicolumn{1}{c|}{R} \\ \hline
1                         & \begin{tabular}[c]{@{}l@{}}Malware:\\ Virus/worm spreads \\ over the Internet \\ possibly affects system \\ files and change \\ firewall settings\end{tabular} & \begin{tabular}[c]{@{}l@{}}Proper maintenance of the \\ server, security patches \\ installed, restricted \\ user rights\end{tabular} & Low                    & Medium                 & Low                    \\ \hline
\end{tabular}
\end{table}


\subsubsection{\it Evaluation Asset Website}

\begin{table}[H]
\centering
\caption{Evaluation Asset Website}
\label{my-label}
\begin{tabular}{|l|l|l|l|l|l|}
\hline
\multicolumn{1}{|c|}{No.} & \multicolumn{1}{c|}{Threat}                                                                                                                                                                                                        & \multicolumn{1}{c|}{\begin{tabular}[c]{@{}c@{}}Implemented/planned \\ countermeasure(s)\end{tabular}}                                                                                 & \multicolumn{1}{c|}{L} & \multicolumn{1}{c|}{I} & \multicolumn{1}{c|}{R} \\ \hline
1                         & \begin{tabular}[c]{@{}l@{}}Skilled hacker gains \\ control\\ over the \\ website, steals confidential \\ data, modifies website \\ settings because of \\ vularability either in \\ the server\\ or on \\ the website\end{tabular} & \begin{tabular}[c]{@{}l@{}}The\\ server is \\ hardened and regularly \\ updated. System \\ administrator is \\ trained to notice \\ irregularities on\\ the \\ server.\end{tabular} & Low                    & Medium                 & Low                    \\ \hline
2                         & \begin{tabular}[c]{@{}l@{}}Script\\ kiddies makes \\ modifications on the\\  website as a result \\ of an attack\end{tabular}                                                                                                      & \begin{tabular}[c]{@{}l@{}}The\\ server is \\ properly maintained, \\ the website is \\ hardened, all input \\ is sanitized\end{tabular}                                              & Low                    & Medium                 & Low                    \\ \hline
\end{tabular}
\end{table}


\subsubsection{\it Evaluation Asset Database}
\begin{table}[H]
\centering
\caption{Evaluation Asset Database}
\label{my-label}
\begin{tabular}{|l|l|l|l|l|l|}
\hline
\multicolumn{1}{|c|}{No.} & \multicolumn{1}{c|}{Threat}                                                                                                                                                                               & \multicolumn{1}{c|}{\begin{tabular}[c]{@{}c@{}}Implemented/planned \\ countermeasure(s)\end{tabular}}                      & \multicolumn{1}{c|}{L} & \multicolumn{1}{c|}{I} & \multicolumn{1}{c|}{R} \\ \hline
1                         & \begin{tabular}[c]{@{}l@{}}Skilled hacker gains \\ control,over the \\ database, steals \\ confidential data, \\ make,changes on \\ the database like \\ deleting tables, \\ editing records\end{tabular} & \begin{tabular}[c]{@{}l@{}}Hardened\\ server and \\ kept up-to-date, \\ use of hashed \\ and salted passwords\end{tabular} & Low                    & High                   & Low                    \\ \hline
\end{tabular}
\end{table}


\subsubsection{\it Evaluation Asset Images}
\begin{table}[H]
\centering
\caption{Evaluation Asset Images}
\label{my-label}
\begin{tabular}{|l|l|l|l|l|l|}
\hline
\multicolumn{1}{|c|}{No.} & \multicolumn{1}{c|}{Threat}                                                                                                            & \multicolumn{1}{c|}{\begin{tabular}[c]{@{}c@{}}Implemented/planned \\ countermeasure(s)\end{tabular}} & \multicolumn{1}{c|}{L} & \multicolumn{1}{c|}{I} & \multicolumn{1}{c|}{R} \\ \hline
1                         & \begin{tabular}[c]{@{}l@{}}Web developer \\ unintentionally breaks \\ confidentiality during \\ the update of the website\end{tabular} & Well trained web developer                                                                            & Low                    & High                   & Low                    \\ \hline
\end{tabular}
\end{table}


\subsubsection{\it Evaluation Asset Username and Password}
\begin{table}[H]
\centering
\caption{Evaluation Asset Username and Password}
\label{my-label}
\begin{tabular}{|l|l|l|l|l|l|}
\hline
\multicolumn{1}{|c|}{No.} & \multicolumn{1}{c|}{Threat}                                                                                   & \multicolumn{1}{c|}{\begin{tabular}[c]{@{}c@{}}Implemented/planned \\ countermeasure(s)\end{tabular}}                       & \multicolumn{1}{c|}{L} & \multicolumn{1}{c|}{I} & \multicolumn{1}{c|}{R} \\ \hline
1                         & \begin{tabular}[c]{@{}l@{}}Script kiddies try to \\ guess the username \\ and the password\end{tabular}       & \begin{tabular}[c]{@{}l@{}}Advising users not \\ to choose simple \\ usernames\\ and passwords\end{tabular}                 & Medium                 & Medium                 & Medium                 \\ \hline
2                         & \begin{tabular}[c]{@{}l@{}}Skilled hacker attacks \\ with special software \\ to break passwords\end{tabular} & \begin{tabular}[c]{@{}l@{}}Encryption\\ passwords \\ with strong hash \\ functions choosing \\ arbitrary salts\end{tabular} & Medium                 & High                   & Medium                 \\ \hline
\end{tabular}
\end{table}

\subsubsection{\it Evaluation Asset Customer confidence}
\begin{table}[H]
\centering
\caption{Evaluation Asset Customer confidence}
\label{my-label}
\begin{tabular}{|l|l|l|l|l|l|}
\hline
\multicolumn{1}{|c|}{No.} & \multicolumn{1}{c|}{Threat}                                                         & \multicolumn{1}{c|}{\begin{tabular}[c]{@{}c@{}}Implemented/planned \\ countermeasure(s)\end{tabular}} & \multicolumn{1}{c|}{L} & \multicolumn{1}{c|}{I} & \multicolumn{1}{c|}{R} \\ \hline
1                         & \begin{tabular}[c]{@{}l@{}}Theft of confidential \\ data\end{tabular}               & \begin{tabular}[c]{@{}l@{}}State-of-the-art \\ security measures,\\ hashing and salting\\ all passwords \\not keeping any sensitive\\ data \end{tabular}                         & Low                    & High                   & Low                    \\ \hline

\end{tabular}
\end{table}





\subsubsection{Risk Acceptance}



\begin{footnotesize}
\begin{prettytablex}{p{2cm}X}
No. of threat & Proposed countermeasure including expected impact  \\
\hline
2.3.5 & Allowing users only 3 login attempts \\
\hline
2.3.5 & using one time password sent by sms \\
\hline
2.3.5 & forcing users to create long passwords (including capital letters,\\& numbers and special characters)\\
\hline
\end{prettytablex}
\end{footnotesize}

\end{document}

%%% Local Variables: 
%%% mode: latex
%%% TeX-master: "../../book"
%%% End: 



\end{document}

%%% Local Variables: 
%%% mode: latex
%%% TeX-master: "../../book"
%%% End: 