\subsection{Implementation}

While looking at the solution, we found some vulnerabilities. \\

\begin{center}
\textbf{WEB SITE}
\end{center}
\begin{itemize}  
\item \textbf{Cleartext submission of passwords (login):}
-passwords are sent over unincrepted connection, this may let someone listening to the network traffic aqcuire the user's password.
this would especially be dangerous if the user uses a public wi-fi.
even if in this case the web service does not contain sensitive data, a lot of people re use the same passwords on different platforms and even for online banking.

\item \textbf{Access to images:}
Anyone can access all the shared images on the fakestagram website on www.fakestagram.com:8080/img/"imagename" even without been logged in, the user has to enter the image name he wants to see,(one can surely guess some easy ones fx: me.jpg or dog.jpg)
this violates the confidentiality requirement


\item \textbf{Cross-site scripting (reflected)} \\

The value of the username request parameter is copied into the HTML document as plain text between tags. The payload <script>alert(1)</script> was submitted in the username parameter. This input was echoed unmodified in the application's response. \\

This proof-of-concept attack demonstrates that it is possible to inject arbitrary JavaScript into the application's response. \\

To solve this issue, a very good way is to validate user input. For example, personal names should consist of alphabetical and a small range of typographical characters, and be relatively short; A year of birth should consist of exactly four numerals; And so on.

\end{itemize}

\begin{center}
\textbf{SYSTEM}
\end{center}

\begin{itemize}
\item \textbf{phpinfo() output accessible}\\

\textbf{Impact}\\
Some of the information that can be gathered from this file includes: The username of the user who installed php, if they are a SUDO user, the IP address of the host, the web server version, the system version(unix / linux), and the root directory of the web server.\\

\textbf{Solution}\\
Delete them or restrict access to the listened files.\\

\item \textbf{php Multiple Vulnerabilities} \\
Installed Version: 5.5.9\\


CVE: CVE-2015-4148, CVE-2015-4147, CVE-2015-2787, CVE-2015-2348, CVE-2015-2331 CVE: CVE-2015-4026, CVE-2015-4025, CVE-2015-4024, CVE-2015-4022, CVE-2015-4021 CVE-2015-3329, CVE-2015-3307, CVE-2015-2783, CVE-2015-1352 CVE-2015-6831, CVE-2015-6832, CVE-2015-6833 CVE-2015-3330\\


\item \textbf{php Multiple Remote Code Execution Vulnerabilities} \\
CVE: CVE-2015-0273, CVE-2014-9705 \\


\item \textbf{php Use-After-Free Remote Code Execution Vulnerability } \\
CVE: CVE-2015-2301\\



\item \textbf{php Use-After-Free Denial Of Service Vulnerability} \\
CVE: CVE-2015-1351\\


\item \textbf{php 'serialize\_function\_call' Function Type Confusion Vulnerability} \\
CVE: CVE-2015-6836\\


\item \textbf{php 'phar\_fix\_filepath' Function Stack Buffer Overflow Vulnerability} \\
CVE: CVE-2015-5590\\


\item \textbf{php Multiple Denial of Service Vulnerabilities} \\
CVE: CVE-2015-7804, CVE-2015-7803\\


\item \textbf{php Out of Bounds Read Memory Corruption Vulnerability} \\
CVE: CVE-2016-1903\\

\item \textbf{Apache HTTP Server Multiple Vulnerabilities} \\
CVE: CVE-2015-3185, CVE-2015-3183\\


\end{itemize}