\subsection{Implementation}

While looking at the solution, we found some vulnerabilities. \\

\begin{center}
\textbf{WEB SITE}
\end{center}
\begin{itemize}  
\item \textbf{Cleartext submission of passwords (login):}
-passwords are sent over unincrepted connection, this may let someone listening to the network traffic aqcuire the user's password.
this would especially be dangerous if the user uses a public wi-fi.
even if in this case the web service does not contain sensitive data, a lot of people re use the same passwords on different platforms and even for online banking.

\item \textbf{Access to images:}
Anyone can access all the shared images on the fakestagram website on www.fakestagram.com:8080/img/"imagename" even without been logged in, the user has to enter the image name he wants to see,(one can surely guess some easy ones fx: me.jpg or dog.jpg)
this violates the confidentiality requirement


\item \textbf{Cross-site scripting (reflected)} \\

The value of the username request parameter is copied into the HTML document as plain text between tags. The payload <script>alert(1)</script> was submitted in the username parameter. This input was echoed unmodified in the application's response. \\

This proof-of-concept attack demonstrates that it is possible to inject arbitrary JavaScript into the application's response. \\

To solve this issue, a very good way is to validate user input. For example, personal names should consist of alphabetical and a small range of typographical characters, and be relatively short; A year of birth should consist of exactly four numerals; And so on.

\item \textbf{DOM based redirection:}\\
The website is vulnerable for DOM-based open redirection. This type of redirection appears when a client-side scripts reads data from a controllable part of DOM like URL and processes it in unsafe way. This vulnerability is used in fishing attacks to force the user to visit malicious sites without realizing it; it opens a door for a hacker to inject malicious code on the page. In the website data is read from document.location and passed to document.location via the following statements: var a = document.location.toString().substr(0,document.location.toString().length-
1)+”:8080/sec”; document.location = a;
The recommendation is input to be validated before redirection.


\item \textbf{Password field with auto-complete enabled::}\\
- /sec/views/login.html\\
- /sec/views/login.html\\

On the website auto-complete function is enabled. It means that the browser can save user's credentials on the machine to retrieve them on later visit of the website. If an attacker gains control over the machine, he/she can retrieve user’s browser-stored credentials. 
Recommendation: In the Form tag or in the relevant INPUT tags auto-complete=”off” could be entered.

\end{itemize}

\begin{center}
\textbf{SYSTEM}
\end{center}

\begin{itemize}
\item \textbf{phpinfo() output accessible}\\


\textbf{Impact}\\
Some of the information that can be gathered from this file includes: The username of the user who installed php, if they are a SUDO user, the IP address of the host, the web server version, the system version(unix / linux), and the root directory of the web server.\\

\textbf{Solution}\\
Delete them or restrict access to the listened files.\\

\item \textbf{php Multiple Vulnerabilities} \\
Installed Version: 5.5.9\\

\textbf{\#1}\\
CVE: CVE-2015-4148, CVE-2015-4147, CVE-2015-2787, CVE-2015-2348, CVE-2015-2331\\
\textbf{Impact} \\
Successfully exploiting this issue allow remote attackers to obtain sensitive information by providing crafted serialized data with an int data type and to execute arbitrary code by providing crafted serialized data with an unexpected data type.

\textbf{Solution}\\
Upgrade to php 5.4.39 or 5.5.23 or 5.6.7 or later. For updates refer to http://www.php.net

\item \textbf{\#2} \\
CVE: CVE-2015-4026, CVE-2015-4025, CVE-2015-4024, CVE-2015-4022, CVE-2015-4021\\

\textbf{Impact} \\
Successfully exploiting this issue allow remote attackers to cause a denial of service , bypass intended extension restrictions and access and execute files or directories with unexpected names via crafted dimensions and remote FTP servers to execute arbitrary code.

\textbf{Solution}\\
Upgrade to php 5.4.41 or 5.5.25 or 5.6.9 or later. For updates refer to http://www.php.net

\item \textbf{\#3} \\
CVE:	CVE-2015-3329, CVE-2015-3307, CVE-2015-2783, CVE-2015-1352\\

\textbf{Impact} \\
Successfully exploiting this issue allow remote attackers to cause a denial of service, to obtain sensitive information from process memory and to execute arbitrary code via crafted dimensions.\\


\textbf{Solution}\\
Upgrade to php 5.4.40 or 5.5.24 or 5.6.8 or later. For updates refer to http://www.php.net\\


\item \textbf{\#4} \\
CVE:	CVE-2015-6831, CVE-2015-6832, CVE-2015-6833\\

\textbf{Impact} \\
Successfully exploiting this issue allow remote attackers to execute arbitrary code and to create or overwrite arbitrary files on the system and this may lead to launch further attacks.\\

\textbf{Solution}\\
Upgrade to php version 5.4.44 or 5.5.28 or 5.6.12 or later. For updates refer to http://www.php.net\\

\item \textbf{\#5} \\
CVE: CVE-2015-3330\\

\textbf{Impact} \\
Successfully exploiting this issue allow remote attackers to cause a denial of service or possibly execute arbitrary code via pipelined HTTP requests.\\


\textbf{Solution}\\
Upgrade to php 5.4.40 or 5.5.24 or 5.6.8 or later. For updates refer to http://www.php.net\\

\item \textbf{php Multiple Remote Code Execution Vulnerabilities} \\
CVE: CVE-2015-0273, CVE-2014-9705 \\

\textbf{Impact} \\
Successfully exploiting this issue allow remote attackers to execute arbitrary code via some crafted dimensions.\\

\textbf{Solution}\\
Upgrade to php 5.4.38 or 5.5.22 or 5.6.6 or later. For updates refer to http://www.php.net\\

\item \textbf{php Use-After-Free Remote Code Execution Vulnerability } \\
CVE: CVE-2015-2301\\

\textbf{Impact} \\
Successfully exploiting this issue allow remote attackers to execute arbitrary code on the target system.\\

\textbf{Solution}\\
Upgrade to php 5.5.22 or 5.6.6 or later. For updates refer to http://www.php.net

\item \textbf{php Use-After-Free Denial Of Service Vulnerability} \\
CVE: CVE-2015-1351\\

\textbf{Impact} \\
Successfully exploiting this issue allow remote attackers to cause a denial of service or possibly have unspecified other impact.\\

\textbf{Solution}\\
Upgrade to php 5.5.22 or 5.6.6 or later. For updates refer to http://www.php.net\\

\item \textbf{php 'serialize\_function\_call' Function Type Confusion Vulnerability} \\
CVE: CVE-2015-6836\\

\textbf{Impact} \\
Successfully exploiting this issue allow remote attackers to execute arbitrary code in the context of the user running the affected application. Failed exploit attempts will likely cause a denial-of-service condition.\\


\textbf{Solution}\\
Upgrade to php version 5.4.45, or 5.5.29, or 5.6.13 or later. For updates refer to http://www.php.net\\

\item \textbf{php 'phar\_fix\_filepath' Function Stack Buffer Overflow Vulnerability} \\
CVE: CVE-2015-5590\\

\textbf{Impact} \\
Successfully exploiting this issue allow remote attackers to execute arbitrary code in the context of the PHP process. Failed exploit attempts will likely crash the webserver.\\

\textbf{Solution}\\
Upgrade to php version 5.4.43, or 5.5.27, or 5.6.11 or later. For updates refer to http://www.php.net\\

\item \textbf{php Multiple Denial of Service Vulnerabilities} \\
CVE: CVE-2015-7804, CVE-2015-7803\\

\textbf{Impact} \\
Successfully exploiting this issue allow remote attackers to cause a denial of service (NULL pointer dereference and application crash).\\

\textbf{Solution}\\
Upgrade to php 5.5.30 or 5.6.14 or later. For updates refer to http://www.php.net\\

\item \textbf{php Out of Bounds Read Memory Corruption Vulnerability} \\
CVE: CVE-2016-1903\\

\textbf{Impact} \\
Successfully exploiting this issue allow remote attackers to obtain sensitive information or cause a denial-of-service condition.\\

\textbf{Solution}\\
Upgrade to php version 5.5.31, or 5.6.17 or 7.0.2 or later. For updates refer to http://www.php.net\\


\item \textbf{Apache HTTP Server Multiple Vulnerabilities} \\
CVE: CVE-2015-3185, CVE-2015-3183\\

\textbf{Impact} \\
Successful exploitation will allow remote attackers to bypass intended access restrictions in opportunistic circumstances and to cause cache poisoning or credential hijacking if an intermediary proxy is in use.\\

\textbf{Solution}\\
Upgrade to version 2.4.14 or later, For updates refer to http://www.apache.org\\
\end{itemize}